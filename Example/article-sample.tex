\documentclass{DIKU-article}[2006/05/09]

\usepackage[utf8]{inputenc}

%\selectdanish % Add this if your report is in Danish

%\setlength{\errorcontextlines}{999} %Can be used for debugging purposes
%\alternativecitationstyle
%\draft

\titlehead{Instructions to use DIKU style files}
\authorhead{Jyrki Katajainen and Kimmo Raatikainen}

\title{Instructions to use {DIKU} style files}

\author{%
Jyrki Katajainen\inst{1}
\and
Kimmo Raatikainen\inst{2}%
}

\institute{%
Department of Computer Science, University of Copenhagen\\
Universitetsparken 5, DK-2100 Copenhagen East, Denmark\\
\email{jyrki@di.ku.dk}
\and
University of Helsinki, Department of Computer Science\\
P.O.~Box 68, FIN-00014 University of Helsinki, Finland\\
\email{Kimmo.Raatikainen@cs.helsinki.fi}%
}

\dates{CPH STL Report 2001-1, August 1994. Revised September 1994,
  October 1994, October 1996, December 2000, April 2001, June 2001,
  February 2005, February 2014.}

\begin{document}

\maketitle

\begin{abstract}
Each report must include an abstract that summarizes the results.
Recommended length is at most 150 words.  The abstract should not
contain any references or displayed equations.  \LaTeX-environment is
\verb|abstract|.
\end{abstract}

\begin{subject}
A report can provide the CR Classification (see the appropriate issue
of \textit{Computing Reviews}), but this is not obligatory.
\LaTeX-environment is \verb|subject|.
\end{subject}

\begin{keywords}
List of keywords (4--6) can be useful but not obligatory.
\LaTeX-environment is \verb|keywords|.
\end{keywords}

\section{Introduction}

We encourage the students to use \LaTeX\ when preparing their
manuscripts. For this purpose we have provided the \LaTeX\ style file
\verb|DIKU-article.cls|, and the Bib\TeX\ style files \verb|DIKU.bst|
and \verb|DIKU-alternative.bst|.  This document describes how these
style files are to be used.

\section{General}

The \LaTeX\ style file \texttt{DIKU-article.cls} does not accept any
options.  In \verb|\documentstyle|-command all options given in
brackets are simply omitted.  The page layout should not be
changed in any way.

Redundant spaces ought to be minimized by careful arrangement of
tables and figures.  Read your \texttt{.log} file carefully.  There
should be no \verb!Overfull \hbox! (as here) and certainly no visible
one (more than \texttt{1pt}).  If necessary, reword the text.  The
preamble command \verb|\draft| can be used to produce a visible
overfullrulebox in the margin.

The style file provides macros to create running heads.  The
\verb|\authorhead| contains the authors' names as ``Firstname
Surname'', ``F. Surname1, F. Surname2'', and ``F. Surname1,
F. Surname2, F. Surname3'' for up to three authors, and ``Surname1
et~al.'' for four or more authors.  The \verb|titlehead| should
contain a short form of the title, not more than 30 characters.

The argument of \verb|\title|-command must be written in sentence case
(capitalize only the first word, proper nouns, and as dictated by
other specific rules).  When the paper has more than one author, the
authors in the argument of \verb|\author|-command are separated by
usual \verb|\and|-command or by \verb|\AND|-command that inserts
vertical glue between the blocks of authors' names. Affiliations
should be given using \verb|\institute|-command. Also here
\verb|\and|-command can be used as in \verb|\author|-command. In a
multi-author paper, to link the authors and their affiliations
\verb|\inst|-command can be used, e.g.~\verb|\inst{1}| produces the
superscript $^1$.

The use of footnotes and appendices should be avoided.  However, if
appendices are necessary, their place is after
\textbf{Acknowledgements} and before \textbf{References}.

\section{Sectioning, numbering, etc.}

\subsection{Sectioning}

The following three \LaTeX-sectioning commands are available:
\verb|\section|, \verb|\subsection|, and \verb|\subsubsection|.
Use also sentence case in all headings.

\subsection{Numbering}
\label{numbering}

The numbering of displayed equations, theorems, figures, tables, and
other ``numbered'' environments follows one of two styles: either
consecutive in each section, or consecutive through the whole paper
(default).  If you prefer the first style, the style file provides
\verb|\twolevelnumbering|-command.  The argument of the command is a
list of environments that are numbered consecutively in each section,
e.g.
\begin{verbatim}
    \twolevelnumbering{figure,equation,theorem}
\end{verbatim}

Tables should be referred to ``Table I''.  Equations and figures
should be referred to in abbreviated forms: ``\eqnref{eqn:example}''
and ``\figref{fig:example}''.  Use macros
\verb|\eqnref{|\emph{eqn-label}\verb|}|,
\verb|\figref{|\emph{fig-label}\verb|}|, and
\verb|\tableref{|\emph{tab-label}\verb|}| to generate the references.

\subsection{Lists of items}

The depths of \LaTeX-environments \verb|itemize| and \verb|enumerate|
are restricted to two.

\subsection{Tabbing environment}

The style file defines \verb|\tabbingstretch| to specify the strut
to be used in the \texttt{tabbing} environment. The functionality of
\verb|\tabbingstretch| is the same as that of \verb|\arraystretch|
in \texttt{array} and \texttt{tabular} environments.

\section{Theorem-like environments}

The style file defines the following theorem-like environments:
\begin{center}
\begin{tabular}{llll}
\verb|theorem|	& \verb|proposition|	& \verb|claim|	& \verb|fact| \\

\verb|lemma|	& \verb|definition|	& \verb|problem|& \verb|remark| \\

\verb|corollary|& \verb|conjecture|	& \verb|example|&  \verb|observation|
\end{tabular}
\end{center}
These are ``numbered'' environments.  The style file also defines the
corresponding ``unnumbered'' environments: \verb|theorem*|, ...,
\verb|observation*|.  The proofs can be typed within the environment
\verb|proof|.

For example, the \LaTeX\ source
\begin{verbatim}
\begin{theorem}[Chebychev's Inequality]
If $X$ is any random variable, then
\begin{equation}
\Pr[|X|\ge a] \le \mbox{\rm E}(X^2)/a^2\;.
\label{eqn:example}
\end{equation}
\end{theorem}
\end{verbatim}
produces
\begin{theorem}[Chebychev's Inequality]
If $X$ is any random variable, then
\begin{equation}
\Pr[|X|\ge a] \le \mbox{\rm E}(X^2)/a^2\;.
\label{eqn:example}
\end{equation}
\end{theorem}

\begin{proof}
If $F(x)$ denotes the distribution function of the random variable
$X$, then
\[
\Pr[|X|\ge a] = \int_{|x|\ge a} \;d\:F(x)\;.
\]
Since in the region of integration $|x|/a\ge1$, it follows that
\[
\int_{|x|\ge a} \;d\:F(x) \le \frac{1}{a^2}
\int_{|x|\ge a} x^2\;d\:F(x)\;.
\]
By extending the integration to all values of $x$, we merely
strengthen the inequality:
\[
\int_{|x|\ge a} \;d\:F(x) \le
\frac{1}{a^2} \int_{|x|\ge a} x^2\;d\:F(x) \le
\frac{1}{a^2} \int x^2\;d\:F(x) = 
\mbox{\rm E}(X^2)/a^2\:.\; \qed
\]
% The following empty line is necessary

\end{proof}

When a proof ends with a displayed equation as above, the box ``\qed''
should be at the right end of the formula rather than at the beginning
of the next line.  In this case, the box must be inserted through
\verb|\qed| at the end of the equation, and a blank line must be left
between the closing of the equation and the \verb|\end{proof}| (as in
the source of these instructions).

\subsection{Spacing before and after environments}

Extra space is added at the top of a list if the input file has a
blank line before any list-making environment.  The vertical space
after the environment is the same as the one preceding it.  The
list-making environments are: \texttt{quote}, \texttt{quotation},
\texttt{verse}, \texttt{itemize}, \texttt{enumerate},
\texttt{description}, \texttt{center}, \texttt{flushleft}, and
\texttt{flushright}, as well as all the theorem-like environments.

\section{Figures and tables}

Figures and tables are to be inserted in the text nearest their first
reference.  They should be arranged so as not to cause an excessive
amount of blank space on the remainder of the page.

The captions are centered below the figures and above the tables.
If a table needs to extend over to a second page, the continuation of the
table should have a caption: ``Table II \emph{(cont.)}''.
Macro \verb|\continued| generates this caption.
Example:
\begin{quote}
   \verb|\begin{table}|\\
   \verb|\continued   % Instead of \caption|\\
   \verb|   ...|\\
   \verb|\end{table}|
\end{quote}

\begin{figure}[tb]
\begin{center}
\input{example.pdf_t}
\caption{This drawing was created with \texttt{xfig} and converted to
  PDF/\LaTeX\ format. Then the \LaTeX\ part was inputted which further
  included the PDF part.}
\label{fig:example}
\end{center}
\end{figure}

The figures could be produced using the \texttt{tikz} package,
\texttt{xfig} vector-graphics editor, any other drawing tool, or
cell-phone camera. The created pictures can be converted to, e.g.~GIF,
JPEG, PNG, or PDF format, and included using any of the graphics
packages available in the \LaTeX\ bundle.  In a high-quality report it
may be necessary to use the same font in the text inside the figure as
that used in the main body of the report.  An example of this is
given in \figref{fig:example}. The picture was drawn with
\texttt{xfig}; all text was written using the special flag and the
default \LaTeX\ fonts.  The picture was exported to two files: a
\LaTeX\ file contained the text and a PDF file contained the graphics.
The \LaTeX\ part is inputted into this file; the \LaTeX\ part further
includes the PDF part.

When including your own pictures, you can use, for example, the
\texttt{graphicx} package, in which case
\begin{quotation}
\verb|\usepackage{graphicx}|
\end{quotation}
must be added to the preamble, after which a new figure can be
inserted by using the command:
\begin{quotation}
\verb|\includegraphics{|\emph{filename}\verb|}|\,.
\end{quotation}
\noindent
For possible options, e.g.~how to scale the picture, consult your
\LaTeX\ documentation.

Remember that \textbf{previously published material must be
  accompanied by written permission from the author and publisher!}

\section{References}

The preferred style of referring to the bibliography is to use
numbered references: ``Raatikainen \cite{Raatikainen-1993} proposed to
use the Bonferroni inequality (see \cite[pp.~41--43]{Kleijnen-1987} or
\cite[\S~9.4]{Law-Kelton-1991}).''  This is obtained by using the
standard \LaTeX-environment \verb|thebibliography| or Bib\TeX\ with
the style file \texttt{DIKU.bst} which generates the bibliography
items in the preferred format.

Another possibility is to use the author-date citation style:
``Raatikainen [1993] proposed to use the Bonferroni inequality (see
[Kleijnen 1987, pp.~41--43] or [Law and Kelton 1991, \S~9.4]).''  Note
that the years are in brackets in running text, but without brackets
if the reference itself is in brackets.  For three or more authors,
use ``et~al.''  Several papers in the same year are distinguished as
``Raatikainen 1994a, Raatikainen 1994b''.  When an article has more
than two authors, the citation should be given in the form ``[Kojo et
  al.\ 1996]''.

The style file provides the command \verb|\alternativecitationstyle|
to support the alternative citation style.
The following macros are defined to simplify the use of the
alternative style:

\begin{center}\begin{tabular}{ll}
\hline
 macro & produces\\ \hline
 \verb|\cite{bibreflabel}| & ``[Author Year]''\\ 
 \verb|\cite[note]{bibreflabel}| & ``[Author Year, note]''\\ 
 \verb|\citet{bibreflabel}| & ``Author [Year]''\\ 
 \verb|\citet[note]{bibreflabel}| & ``Author [Year, note]''\\ 
 \verb|\citealt{bibreflabel}| & ``Author Year''\\ 
 \verb|\citealt[note]{bibreflabel}| & ``Author Year, note''\\ 
 \verb|\citeauthor{bibreflabel}| & ``Author''\\ 
 \verb|\citeyear{bibreflabel}| & ``Year''\\ 
\hline
\end{tabular}\end{center}

\noindent
The macros above assume that your bibliography items are written as
\begin{verbatim}
   \bibitem[{Author}{Year}]{bibreflabel}
   \bibitem[{Author}{Year1}]{bibreflabel1}
   \bibitem[{Author}{Year2}]{bibreflabel2}
\end{verbatim}
The Bib\TeX\ style file \texttt{DIKU-alternative.bst} generates
the bibliography items in the format above.

\section{Use of the style files}

We provide the following files:
\begin{enumerate}
\item
\texttt{DIKU-article.cls} contains the \LaTeX\ code for producing the
camera-ready output of your report.

\item 
\texttt{DIKU-report.cls} is designed for longer reports.
It is used in the same way as \texttt{DIKU-article.cls}, but it
provides some additional features like the \verb|titlepage|
environment, commands \verb|\frontmatter|,
\verb|\mainmatter|, \verb|\backmatter|, \verb|\tableofcontents|,
\verb|\chapter|, \verb|\part|, and some undocumented features (for
these you have to look at the \LaTeX\ source). 
\item 
\texttt{DIKU.bst} contains the Bib\TeX\ code which produces a reference
list in the form preferred by us.

\item
\texttt{DIKU-alternative.bst} contains the Bib\TeX\ code which produces a reference list
in the form suitable for the author-year citation style.
\item
\texttt{article-sample.tex} contains the \LaTeX\ source of these
instructions.
\item
\texttt{article-sample.bib} contains the Bib\TeX\ source to produce
the reference list.

\item \texttt{article-sample.dict} is created by \texttt{ispell}.
  It contains correctly spelled words that are not in the standard
  dictionary.
\item
\texttt{example.fig} contains the example figure in
 \texttt{xfig} format.

\item
\texttt{thesis-sample.tex} contains a brief introduction (in Danish)
how to use the style file \verb|ku-forside.sty| when writing a thesis
using \verb|DIKU-report.cls|.
\item 
\texttt{makefile} is provided to make your life easier.
\item \texttt{article-sample.www} contains data used by our
  crawler. When this file exists, the report will be made visible via
  our web pages at \verb|cphstl.dk|.
\end{enumerate}

\noindent
To prepare your report, type:
\begin{verbatim}
   shell> pdflatex article-sample.tex
   shell> bibtex article-sample
   shell> pdflatex article-sample.tex
   shell> pdflatex article-sample.tex
\end{verbatim}
or
\begin{verbatim}
   shell> make
\end{verbatim}
 
\section{Disclaimer}

The style files are not guaranteed to be free of errors.  Any bugs,
inconsistencies, suggestions, and other comments should be reported by
email to \texttt{jyrki@di.ku.dk}.

\begin{acknowledgements}

This section comes before the references and is unnumbered.
\LaTeX-en\-viron\-ment is \verb|acknowledgements|.

This style file was created by modifying the style file
\texttt{njcarticle.cls} provided by the \textit{Nordic Journal of
  Computing} (NJC) and copyrighted by Kimmo Raatikainen. He gave us
the permission to use the style on 29 April 1996. His instructions for
the NJC authors were used as the first draft of this document.
\end{acknowledgements}

\nocite{mowgli-overview,Raatikainen-1994a,Raatikainen-1994b}
\bibliographystyle{DIKU} % Use DIKU-alternative for the other citation style
\bibliography{article-sample}

\end{document}

