\documentclass[xetex,mathserif,serif]{beamer}

\usepackage{amsmath}
\usepackage{graphicx}
\usepackage{bm}
\usepackage{multicol}
\usepackage{xcolor}

\usepackage{tikz}
\usetikzlibrary{decorations.markings, arrows}
\usepackage{tikz-cd}

\newcommand\mrm[1]{\mathrm{#1}}
\newcommand\msc[1]{\normalfont\textsc{#1}}
\newcommand\brm[1]{\bm{\mrm{#1}}}
\newcommand\Nat{\mathbb{N}}
\newcommand\Zheil{\mathbb{Z}}
\newcommand\Real{\mathbb{R}}
\newcommand\Com{\mathbb{C}}
\newcommand\Arr[1]{{\brm{Arr}_{\brm{#1}}}}
\newcommand\SFR{\sharp\flat\varrho}
\newcommand\SSFR{\sigma\sharp\flat\varrho}
\newcommand\DSSFR{\div\sigma\sharp\flat\varrho}
\newcommand\ADSSFR{\approx\div\sigma\sharp\flat\varrho}
\newcommand\CADSSFR{c\cdot\approx\div\sigma\sharp\flat\varrho}
\newcommand\XCADSSFR{\times c\cdot\approx\div\sigma\sharp\flat\varrho}
\newcommand\impl{\mathrel{\Rightarrow}}
\newcommand\tlaf{\mathop{\rotatebox[origin=c]{180}{$\flat$}}}
\newcommand\reduce{\operatorname*{\brm{reduce}\,}}
\newcommand\diag{\operatorname*{\brm{diag}\,}}
\newcommand\tile{\operatorname*{\brm{tile}\,}}
\newcommand\oprodby[1]{\mathop{\operatorname*{\,\brm{by}}_{#1}\,}}

\newcommand\pto{\mathrel{\ooalign{\hfil$\mapstochar\mkern5mu$\hfil\cr$\to$\cr}}}

\newcommand\NB[1]{\textcolor{red}{#1}}

\begin{document}

\begin{frame}
  \begin{center}
    \usebeamercolor[fg]{title}
    {\Large Multidimensional Arrays: Formally Specified by Shape Description and Sliced by Affine Functions}\\[1em]
    Bachelor's thesis by Karl D. Asmussen \\
    Thesis advisor Jyrki Katajainen
  \end{center}
\end{frame}


\begin{frame}
  \frametitle{Core definitions}
  \pause
  \begin{itemize}[<+->]
    \item Array : integer coordinates $\pto$ set of entries
    \item Rank is \#dimensions
    \item Shape is bounding box corner
    \item Indexing is function application
  \end{itemize}
\end{frame}

\begin{frame}
  \frametitle{Core definitions --- Example}
  \pause
  \begin{tabular}{lll}
    \vspace{0.2em}
    Array &
    \(\brm A = [0\ 100\ 200\ 300]\) &
    \(\brm B = \begin{bmatrix}
        00 & 01 & 02 \\
        10 & 11 & 12 \\
        20 & 21 & 22
      \end{bmatrix}\) \pause \\
      Shape &
      \(\sharp\brm A = [4]\) & \(\sharp\brm B = [3\ 3]\) \pause \\
      Rank &
      \(\sharp\sharp\brm A = [1]\) & \(\sharp\sharp\brm B = [2]\) \pause \\
      Index &
      \(\brm A(2) = 200\) & \(\brm B(1, 2) = 12\)
  \end{tabular}
\end{frame}

\begin{frame}
  \frametitle{Thesis}
  \pause
  \begin{itemize}[<+->]
    \item Operations on shape lists $\iff$ operations on arrays
    \item Operations on shape products $\iff$ operations on arrays
    \item Deep theory $\implies$ powerful understanding
    \item Extant: Ad hoc
  \end{itemize}
\end{frame}

\begin{frame}
  \frametitle{Minimal set of operations}
  \pause
  \begin{multicols}{3}
    \begin{itemize}[<+->]
      \item Transpose
      \item Diagonal
      \item Tile
      \item Fixate
      \item Slice
      \item Reverse $^*$
      \columnbreak
      
      \item Upshape $^*$
      \item Flat
      \item Reshape
      \vfill
      \item Map
      \columnbreak

      \item Reduce
      \item Outer product
      \item Concatenate $^*$
      \item Subarray $^*$
      \vfill
    \end{itemize}
  \end{multicols}
\end{frame}

\begin{frame}
  \frametitle{Inspiration}
  \pause
  \begin{multicols}{3}
    \footnotesize
    \begin{itemize}[<+->]
      \item Commutativity, permutation
      \item Square-free, remove duplicates
      \item Scale, insert
      \item Delete, divide $^*$
      \item Vector subtraction
      \item ---
      \columnbreak
      
      \item Factorization
      \item Multiplication
      \item Equality u/ mult.
      \vfill
      \item Identity
      \columnbreak

      \item Deltete, divide
      \item Concatenate, multiply
      \item Tensor addition
      \item Commutativity, subsequence extraction
      \vfill
    \end{itemize}
  \end{multicols}
\end{frame}

\begin{frame}
  \frametitle{Pre-/postcompositional operations}
  \pause
  \begin{itemize}[<+->]
    \item Precompose:
    \begin{itemize}[<+->]
      \item Linear transform: Transpose, Diagonal, Tile, Upshape$^*$
      \item Homogenous/Affine transform: Fixate, Reverse, Slice
      \item Monotonous map: Upshape, Flat, Reshape
    \end{itemize}
    \item Postcompose: Map
  \end{itemize}
\end{frame}

\begin{frame}
  \frametitle{Higher-order operations}
  \pause
  \begin{itemize}[<+->]
    \item Outer product --- categorical product
    \item Reduce --- fold
    \item Subarray --- maybe exponential?
    \item Concatenate --- odd-one-out
  \end{itemize}
\end{frame}

\tikzset{
  -|->/.style={
    decoration={
      markings,
      mark=at position #1 with {\arrow{|}},
      mark=at position 1.0 with {\arrow{>}}
    },
    postaction={decorate}
  },
  -|->/.default=0.5 
}

\begin{frame}
  \frametitle{Category theory}
  \pause
  \begin{center}
    \begin{tikzcd}[ampersand replacement=\&]
      \Zheil^d \ar[-|->, r, "A"] \ar[d, "\alpha", swap] \& S \ar[d, "\omega"] \\
      \Zheil^g \ar[-|->, r, "B", swap] \& T 
    \end{tikzcd}\\[2em]
    \pause

    \begin{tikzcd}[ampersand replacement=\&]
      \Zheil^d \ar[-|->, r, "A"{name=UPPER}] \& S \\
      \Zheil^g \ar[-|->, r, "B"{name=LOWER}, swap] \& T
      \ar[-implies, from=UPPER, to=LOWER, double, shorten <= 5pt, shorten >= 5pt, "\Theta"]
    \end{tikzcd}
  \end{center}
\end{frame}

\begin{frame}
  \frametitle{Preservation of origin entry}
  \pause
  \begin{multicols}{2}
    \begin{itemize}[<+->]
      \item Transpose
      \item Diagonal
      \item Tile
      \item Upshape
      \item Flat
      \item Reshape
      \vfill
      \columnbreak

      \item Map
      \item Reduce
      \item Outer product
      \item Subarray\\[2em]
      \item Concatenate
    \end{itemize}
  \end{multicols}
\end{frame}

\begin{frame}
  \frametitle{Implementation}
  \pause
  \begin{multicols}{2}
  \begin{itemize}[<+->]
    \item Rust --- ML
    \item NumPy --- general
    \item Owned/Reference
    \item Reference/Mutable
    \item Lifetimes
    \item $\neg$HKP
    \columnbreak

    \item Big traits
    \item Boilerplate
    \item Perlisism \#15
    \item Compiles, but unsound
    \item v0.9$\alpha$ to v1.0$\beta$
  \end{itemize}
  \end{multicols}
\end{frame}

\begin{frame}
  \frametitle{}
  \begin{itemize}
    \item Work in progress
  \end{itemize}
\end{frame}

\end{document}
