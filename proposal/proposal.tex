\documentclass[a4paper,11pt,notitlepage]{article}

\usepackage[english]{babel}
\usepackage[utf8]{inputenc}
\usepackage{fontspec}
\usepackage{titling}
\usepackage{abstract}
\usepackage[margin=1in,headheight=3em]{geometry}
\usepackage{fancyhdr}
\usepackage{graphicx}
\usepackage{tikz}
\usepackage{dot2texi}
\usepackage{placeins}
\usepackage{lastpage}
\usepackage{titlesec}
%\usepackage{tocloft}
%\usepackage[tocflat]{tocstyle}
\usepackage{xcolor}
\usepackage[linktocpage=true]{hyperref}
\usepackage{amsmath}
\usepackage{amssymb}
\usepackage{listings}
\usepackage[english]{isodate}
\usepackage[backend=biber]{biblatex}
\usepackage{makeidx}
\usepackage{bm}

\setmainfont{Times New Roman}
\setsansfont{Arial}
\setmonofont[Scale=0.8]{Fira Mono}

\newfontfamily\apltt{APL385 Unicode}
\DeclareTextFontCommand\textapl{\apltt}
\newfontfamily\ttlig{Fira Code}
\DeclareTextFontCommand\textttlig{\ttlig}

\makeatletter
\newcommand{\verbatimfont}[1]{\def\verbatim@font{#1}}%
\makeatother
\verbatimfont{\ttfamily}

\lstdefinelanguage{Rust}{
  morekeywords={abstract, alignof, as, become, box, break, const, continue, crate, do, else, enum, extern, false, final, fn, for, if, impl, in, let, loop, macro, match, mod, move, mut, offsetof, override, priv, proc, pub, pure, ref, return, Self, self, sizeof, static, struct, super, trait, true, type, typeof, unsafe, unsized, use, virtual, where, while, yield},
  sensitive=true,
  morecomment=[l]{//},
  morecomment=[n]{/*}{*/},
  morestring=[b]",
  morestring=[b]',
}
\lstdefinestyle{RustPrelude}{
  emph={Copy, Send, Sized, Sync,
  Drop, Fn, FnMut, FnOnce,
  Box,
  ToOwned,
  Clone,
  PartialEq, PartialOrd, Eq, Ord,
  AsRef, AsMut, Into, From,
  Default,
  Iterator, Extend, IntoIterator, DoubleEndedIterator, ExactSizeIterator,
  Option, Some, None,
  Result, Ok, Err,
  SliceConcatExt,
  String, str, ToString,
  Vec},
}
\lstdefinelanguage[x86_64gas]{Assembler}[x86masm]{Assembler}{
  alsoletter={.\%},
  alsodigit={\$},
  deletecomment=[l];,
  deletestring=[b]',
  morecomment=[l]\#,
  morecomment=[s]{/*}{*/},
  deletekeywords=[2]{
    .align, .alpha, assume, byte, code, comm, comment, .const, 
    .cref, .data, .data?, db, dd, df, dosseg, dq, dt, dw, dword, else, end, endif, 
    endm, endp, ends, eq, equ, .err, .err1, .err2, .errb, .errdef, .errdif, 
    .erre, .erridn, .errnb, .errndef, .errnz, event, exitm, extrn, far, 
    .fardata, .fardata?, fword, ge, group, gt, high, if, if1, if2, ifb, ifdef, 
    ifdif, ife, ifidn, ifnb, ifndef, include, includelib, irp, irpc, label, 
    .lall, le, length, .lfcond, .list, local, low, lt, macro, mask, mod, .model, 
    name, ne, near, offset, org, out, page, proc, ptr, public, purge, qword, .
    radix, record, rept, .sall, seg, segment, .seq, .sfcond, short, size, 
    .stack, struc, subttl, tbyte, .tfcond, this, title, type, .type, width, 
    word, .xall, .xcref, .xlist
  },
  morekeywords=[2]{
    .abort, .align, .app-file, .ascii, .asciz, .balign, .byte, .comm,
    .data, .def, .desc, .dim, .double, .eject, .else, .endef, .endif, .endm,
    .equ, .equiv, .err, .extern, .file, .fill, .float, .global, .globl,
    .hword, .ident, .if, .include, .int, .irp, .irpc, .lcomm, .lflages,
    .line, .linkonce, .ln, .mri, .list, .long, .macro, .nolist, .octa, 
    .org, .p2align, .psize, .quad, .rept, .sbttl, .scl, .section, .set,
    .short, .single, .size, .sleb128, .skip, .spacce, .stabd, .stabn, .stabs,
    .string, .symver, .tag, .text, .title, .type, .val, .uleb128, .word
  },
  morekeywords={
    CDQE, CQO, CMPSQ, CMPXCHG16B, IRETQ, JRCXZ, LODSQ, MOVSXD, POPFQ, PUSHFQ, RDTSCP, SCASQ, STOSQ, SWAPGS,
    MOVQ, ADDQ, SUBQ, CMPQ, TESTQ,
    rax, rbx, rcx, rdx, rsp, rbp, rsi, rdi, rip,
     r8,  r8d,  r8w,  r8b, 
     r9,  r9d,  r9w,  r9b, 
    r10, r10d, r10w, r10b, 
    r11, r11d, r11w, r11b, 
    r12, r12d, r12w, r12b, 
    r13, r13d, r13w, r13b, 
    r14, r14d, r14w, r14b, 
    r15, r15d, r15w, r15b, 
  }
}
\lstdefinestyle{x86_64}{
  language=[x86_64gas]Assembler,
  style=Normal,
}
\lstdefinelanguage{APL}{
  morestring=[b]',
  morecomment=[l]⍝,
  otherkeyword={⌶,⌷,⌸,⌹,⌺,⌻,⌼,⌽,⌾,⌿,⍀,⍁,⍂,⍃,⍄,⍅,⍆,⍇,⍈,⍉,⍊,⍋,⍌,⍍,⍎,⍏,⍐,⍑,⍒,⍓,⍔,⍕,⍖,⍗,⍘,⍙,⍚,⍛,⍜,⍞,⍟,⍠,⍡,⍢,⍣,⍤,⍥,⍦,⍧,⍨,⍩,⍪,⍫,⍬,⍭,⍮,⍯,⍰,⍱,⍲,⍳,⍴,⍵,⍶,⍷,⍸,⍹,⍺,⎕},
}
\lstdefinestyle{APL385}{
  basicstyle=\footnotesize\apltt,
}
\lstdefinestyle{Normal}{
  frame=single,
  basicstyle=\footnotesize\ttfamily,
  numbers=left,
  numbersep=5pt,
  tabsize=2,
  showstringspaces=false, 
  keywordstyle=\color{blue},
  stringstyle=\color{orange!50!black},
  emphstyle=\color{green!50!black},
}
\lstdefinestyle{Rust}{
  language=Rust,
  style=Normal,
  style=RustPrelude,
}
\lstdefinestyle{APL}{
  language=APL,
  style=Normal,
  style=APL385,
}

\usetikzlibrary{shapes,arrows}

\makeindex

\setlength\parindent{0pt}
\setlength\parskip{0.5em}
\setlength\absparindent{0em}
\setlength\absparsep{0.5em}

\titlespacing*{\section}{0pt}{2\baselineskip}{1\baselineskip}
\titlespacing*{\subsection}{0pt}{1.5\baselineskip}{0.5\baselineskip}
\titlespacing*{\subsubsection}{0pt}{1\baselineskip}{0.25\baselineskip}

\makeatletter
\renewcommand{\l@section}{\bfseries\@dottedtocline{1}{0em}{0em}\mdseries}
\renewcommand{\l@subsection}{\@dottedtocline{2}{2em}{0em}}
\renewcommand{\l@subsubsection}{\itshape\@dottedtocline{3}{4em}{0em}\upshape}
\makeatother

\isodate
\isodash{--}

\hypersetup{
  colorlinks,
  linkcolor={red!50!black},
  citecolor={blue!50!black},
  urlcolor={blue!80!black}
}

\pagestyle{fancyplain}
\fancyhf{}
\renewcommand\headrulewidth{0pt}

\makeatletter
\fancyhf[HR]{%
\@author~\\ %
\@title%
}

\fancyhf[FL]{%
\@date%
}
\makeatother

\fancyhf[FR]{%
\thepage\ / \pageref{LastPage}%
}

\newcommand\First[1]{%
%\renewcommand\thesection{\arabic{section}}%
\renewcommand\thesection{}%
%\renewcommand\thesubsection{\arabic{section}.\arabic{subsection}}%
\renewcommand\thesubsection{}%
%\renewcommand\thesubsubsection{\arabic{section}.\arabic{subsection}.\arabic{subsubsection}}%
\renewcommand\thesubsubsection{}%
\maketitle%
#1%
\tableofcontents%
}

\newcommand\Middle[1]{%
\vfill%
#1%
\clearpage%
\appendix%
\phantomsection%
\addcontentsline{toc}{section}{Appendices}%
%\renewcommand\thesection{\arabic{section}}%
\renewcommand\thesection{}%
%\renewcommand\thesubsection{\arabic{section}.\arabic{subsection}}%
\renewcommand\thesubsection{}%
%\renewcommand\thesubsubsection{\arabic{section}.\arabic{subsection}.\arabic{subsubsection}}%
\renewcommand\thesubsubsection{}%
\fancyhf[HL]{appendix}%
}

\newcommand\Last[1]{%
\phantomsection%
\addcontentsline{toc}{section}{References, Lists \& Indexes}%
\printbibliography%
\ifnum\value{figure}>0\listoffigures\fi%
\ifnum\value{table}>0\listoftables\fi%
\printindex%
#1%
}

\newcommand\brm[1]{\bm{\mathrm{#1}}}


\author{Karl D. Asmussen (\texttt{twk181}) at DIKU}
\addbibresource{refs.bib}

\fancyhf[HL]{bachelor thesis proposal}
\title{Formalization of \(N\)-dimensional Arrays, Implementation in Rust}

\begin{document}
\First

\section{Purpose}

To attempt to formally ground classification of operations on \(N\)-dimensional arrays in category
theory, in order to hopefully expose new insights into \(N\)-dimensional arrays,
their operations, optimizations, storage, and handling.

To implement a lightweight \(N\)-dimensional array library in Rust\footnote{Most recent
stable version of \texttt{rustc} and official standard library as of \today, is 1.15.1.}
using the idiom of dual data types to handle the `owned' heap allocation and the `borrowed' slice,
focusing on a very expressive slice type.

\section{Motivation}

\(N\)-dimensional arrays are an important concept in numerical computing, used in
a wide variety of applications, often in the implementation of numerical algorithms and
linear algebra operations.

Popular and powerful implementations of \(N\)-dimensional arrays are found in Python's NumPy,
Matlab, GNU Octave, and Mathematica; offering a host of functionality and easy/abstracted interfaces.
These all have a clear advantage over the `old-school' way of programming numerical algorithms
by hand in C or FORTRAN.

Unfortunately, to my knowledge, very little work has been done to formalize the operations
on multidimensional arrays beyond the specialized applications of Tensors, and the two-dimensional
matrices of linear algebra. Most libraries are designed from strictly practical standpoints, including
functionality at such a time as it becomes pertinent to the user base, or more generally as the author's
intuition seems to require it.

The most abstract work I have encountered regarding the semantics of \(N\)-dimensional arrays, is the
explanation of broadcasting rules in NumPy\cite{scipy17}. It goes in depth with a rather convenient format
for interpreting how scalars interact in multidimensional contexts, as well as how lower-dimensional arrays
can `stand in' for higher dimensions.

What I aim to do, is provide a theory that explains why broadcasting is useful and makes sense.

The choice of programming language is owed to Rust's excellent polymorphism and memory safety
features, making it easy to work with array slices without fear of accessing invalid memory. In
simple terms, because of Rust's unique static lifetime analysis, it is impossible for a
reference to outlive what it refers to, without relying on garbage collection.
(Also, data-races are impossible as well.)

Because of the safe reference semantics, Rust's standard library has an idiom of each
array-like collection having a slice type associated. See Table~\ref{tab:slice-types}

\begin{table}
  \begin{center}
    \begin{minipage}{0.8\textwidth}
      \begin{center}
        \ttfamily
        \begin{tabular}{r | l}
          \rmfamily Allocation type & \rmfamily Slice type \\\hline
          Vec<T> & \&[T] \\
          Box<T> & \&[T] \\
          String & \&str \\
          PathBuf & \&Path \\
          OsString & \&OsStr \\
          OsString & \&OsStr \\
          CString & \&CStr \\
        \end{tabular}
      \end{center}
      \caption[Rust slice type idiom]{Illustration of the rust slice type idiom. The \texttt{\&} denotes
      a reference type (non-arithmetic, non-null pointer.)}
      \label{tab:slice-types}
    \end{minipage}
  \end{center}
\end{table}

My primary inspirations for working on this project is Python's NumPy library and
the APL-programming language family. NumPy provides a model for translating the rather
obscure functionality of APL into modern programming languages, while APL long ago broke
much new ground in what kinds of array operations are possible.\footnote{Disclaimer: I do not wish to encroach on the territory of obtusely terse, non-ASCII
syntax.}

\section{Method}

Theoretically, I'm looking at modeling multidimensional arrays as something akin to a
free functor construction in category theory. This functor, essentially, takes a set to
the subcategory of multidimensional arrays over elements of that set, and it takes a
function to the common map-operation.

The unexplored territory is then to observe what operations on multidimensional arrays map
to the identity function in Set. In other words, the operations that one can perform on an
array, independently of its type --- the manipulations of arrays characterized by e.g. transposition
of matrices, slicing, replication, and so forth.

To characterise these manipulations, I wish to establish a correspondence between the
array operations, and the resulting effect on the shape-list of the arrays.

Practically, I hope to implement multidimensional arrays as a polymorphic type in Rust,
and work with making a powerful array/array-slice type pair, in lieu of the Rust idiom of
having allocated `owned' types, and light `slice' types. The more operation can be embedded
in the slice type, the less allocations are needed.

Here, I'm hoping to work with an idea of modeling slices with a list of the bounds on
each index, and an affine functional to transform a list of indexes into an index into a flat
array.

Additionally, it may be prudent to work with lazy constructions to avoid computing enormous
outer products when one only wants the diagonal (see motivating example below.)

\subsection{Motivating theoretical example: matrix product}

The product of two matrices \(\brm A, \brm B\) exists only when \(\brm A\) has as many columns
as \(\brm B\) has rows. Matrices are (often times) 2-dimensional arrays of real numbers,
so we describe their shape lists as \(\brm A : n \times m, \; \brm B : m \times p\).

The product will then be \(\brm A \brm B = \brm C : n \times p\).

Computationally this is done by combining every row of \(\brm A\) with every column
of \(\brm B\) by way of the inner product on vectors.

But focusing on the shapes of the matrices, we combined \(n \times m\) and \(m \times p\) and
got \(n \times p\), which seems to skip a few steps.

I posit that there are in fact three steps involved: an operation similar to the outer product,
extracting a diagonal, and a fold-operation.

The outer product of two vectors is the combination of every component in the first vector,
by way of salar multiplication, with the second vector (or vice versa.) For convenience,
the resulting vectors are stored as the rows (or columns) of a matrix. The outer product
of two 1-dimensional arrays is thus a 2-dimensional array.

The outer product of two matrices is the combination of every entry of the first matrix, by
way of scalar multiplication, with the second (or vice versa.) Thus we get a grid of matrices:

\begin{align*}
  \brm A \otimes \brm B = \begin{bmatrix}
    a_{11} \brm B & \cdots & a_{1m} \brm B \\
    \vdots & \ddots & \vdots \\
    a_{n1} \brm B & \cdots & a_{nm} \brm B \\
  \end{bmatrix} = \begin{bmatrix}
    \brm A b_{11} & \cdots & \brm A b_{1p} \\
    \vdots & \ddots & \vdots \\
    \brm A b_{m1} & \cdots & \brm A b_{mp} \\
  \end{bmatrix}
\end{align*}

and it is obvious that the outer product of two matrices is a 4-dimensional array and
that this has the shape: \(\brm A \otimes \brm B : n \times m \times m \times p\)

The diagonal of a matrix \(\brm M : n \times n\) is the sequence \(M_1 = m_{11}, M_2 = m_{22}, \dots, M_n = m_{nn}\).
Only square matrices has `proper' diagonals, and one achieves the diagonal by traversing the two indexes into the
matrix (row, column) in lockstep. This reduces the two-dimensional \(\brm M : n \times n\) to \(M : n\).

Applying this operation to the outer product gives us \(\operatorname{diagonal}\brm A \otimes \brm B : n \times m \times p\).
Notice that \((\operatorname{diagonal} \brm A \otimes \brm B)_{ijk} = a_{ij} b_{jk}\), which looks very similar to the
Einstein notation for matrix products.

Indeed, the last operaiton is to remove the middle index, \(m\), by way of summing over it. This operation is
commonly known as the `reduce' of the map-reduce algorithm fame, and is available in many libraries to do with
lists. This leads to the final expression of the matrix product as
\((\brm A \brm B)_{ik} = \sum_{j} (\operatorname{diagonal} \brm A \otimes \brm B)_{ijk}\). 

This breaks the otherwise fundamental matrix product up into three `atomic' operations. Notice even,
that the extract-diagonal-and-sum operation if used on a square matrix is exactly the trace of said matrix.

\subsection{Motivating practical example: memory addressing in X86 assembly}

The memory addressing arithmetic found in the X86 assembly language, and prominently
accessible as a first-class arithmetic operation in the \texttt{lea} instruction family,
provides a minimal example for using an affine functional to index multidimensional arrays.

Recall that a functional is a linear function from a vector space to scalars. These exactly mirror
vectors, and indeed can be identified as the functions one get from taking the inner product with
a specific vector:

\begin{align*}
  f_{\brm v}(\brm x) = \brm v \cdot \brm x
\end{align*}

The set of roots in a linear functional form a hyperplane in the vector space, which includes the origin,
and is perpendicular to the vector used as referece.

An affine linear functional is a linear functional that adds a constant as well:

\begin{align*}
  f_{\brm v}(\brm x) = \brm v \cdot \brm x + b =
  \begin{bmatrix} \brm v \\ b \end{bmatrix}\cdot\begin{bmatrix} \brm x \\ 1 \end{bmatrix}
\end{align*}

The \texttt{lea} instruction allows for three parameters: a base, an index, and a constant scalar,
(in the GNU syntax):

\begin{lstlisting}[style=x86_64]
        leaq (%rbp, %rdx, 4), %rsi # rsi = rbp + (4 * rdx)
\end{lstlisting}

\texttt{lea} takes three parameters; an array register, an index register,
and a constant scale. The scale can be one of 1, 2, 4, and 8. In the above example the
base pointer (\texttt{rbp}) serves as the array, the data register (\texttt{rdx}) serves
as the index, and the scale is 4.\footnote{Technically, one can also add a constant, but
this is less relevant in my theory.}

Thus, we are actually computing the following inner product:

\begin{align*}
  rsi = \begin{bmatrix} 4 \\ rbp \end{bmatrix} \cdot \begin{bmatrix} rdx \\ 1 \end{bmatrix}
\end{align*}

This is a poignant example of how an affine functionals are already used to compute array
indexes, and serves as motivation to extend this idea.

Things to consider is, that we want each \(N\)-dimensional index to uniquely correspond to
a flat index. Ergo, the scales for each index need to have the nature of a multiple-radix number
system.

In the base-10 number system, each digit has the values \(0\dots9\) and each position's multiplier
scales by 10. In a multiple-radix number system, each digit has its own radix, and each position has
its multiplier scale by that radix.

This means that the number \(825_{10,4,9}\) is \(5 \times 1 + 2 \times 9 + 8 \times (9 \times 4) = 331_{10}\) ---
the scale for a given digit is the product of all preceding radices.

Ensuring that this property holds becomes tricky if one allows swapping the order of indexes. Swapping
indexes is essentially a generalization of the matrix transpose.

\section{Expected Result}

Ideally, I will arrive at a formal description of the category of multidimensional arrays over sets.
This will require finding an appropriate formalism for the `allowed' and `disallowed' intuitive notions
of array operations.

I also wish to put together a usable, if bare-bones multidimensional array library in Rust which utilizes the
formalism, and the affine functional model of multi-to-single index conversion to implement as much functionality
as possible in the `slice' layer.

\Middle

\Last
\end{document}
