\documentclass{DIKU-report-variant}

\usepackage[titelside, nat, farve]{ku-forside}
\opgave{Bachelorproject}
\forfatter{Karl D. Asmussen}
\titel{Multidimensional Arrays}
\undertitel{Formally Specified and Cleverly Sliced}
\vejleder{Jyrki Katajainen}
\dato{\today}

\titlehead{Multidimensional Arrays: Formally Specified and Cleverly Sliced}
\authorhead{Karl D. Asmussen}

\dates{March 2017}

\usepackage[backend=biber,]{biblatex}
\usepackage{makeidx}
\addbibresource{refs.bib}
\makeindex

\usepackage{fontspec}
\usepackage[english]{babel}

\usepackage{color}
\usepackage{xcolor}
\usepackage{listings}
\usepackage[linktocpage=true]{hyperref}
\usepackage{graphicx}
\usepackage{tikz}
\usetikzlibrary{shapes,arrows}

\usepackage{amsmath}
\usepackage{amssymb}
\usepackage{wasysym}
\usepackage{bm}

\twolevelnumbering{lemma}
\twolevelnumbering{definition}
\twolevelnumbering{remark}
\twolevelnumbering{theorem}
\twolevelnumbering{observation}
\twolevelnumbering{example}

\newcommand\brm[1]{\bm{\mathrm{#1}}}
\let\mrm\mathrm
\newcommand\Nat{\mathbb{N}}
\newcommand\Real{\mathbb{R}}
\newcommand\Com{\mathbb{C}}
\newcommand\sharpeq{\stackrel{\sharp}{=}}
\newcommand\prodeq{\stackrel{\times}{=}}
\newcommand\flateq{\stackrel{\flat}{=}}
\newcommand\then{\mathrel{\Rightarrow}}
\newcommand\contradict{\def\qed{\hspace*{\fill}\raisebox{-.2ex}{\lightning}}}

\begin{document}
\maketitle

\begin{abstract}
  % overview
  We attempt to formalise multidimensional arrays in a category theoretical
  setting and provide an example library implementation in Rust.
  % specific problem
  Many libraries exist to provide support for multidimensional data-sets, but all
  of them are based on ad-hoc premises.
  % review of existing solutions
  Library implementations seen in NumPy, Matlab, Mathematica and R, are all
  perfectly servicable, but all function on different semantics.
  % outline of solution
  By formalizing operations on multidimensional arrays, we may be able to
  describe the differing semantics of other libraries.
  % evalutation parameters
  In the end, we stress test knowledge and implementation, by applying it to
  a bachelor-level statistics problem and implementing a linear algebra algorithm.
\end{abstract}

\begin{keywords}
  Data Structure, N-d Array, Numerical Computing, Rust, NumPy
\end{keywords}

\setcounter{tocdepth}{2}
\tableofcontents

\chapter{Formalization}

\section{Motivation}

Arrays-of-arrays are a ubiquitous data structure in computer vision,
computer graphics, numerical computation. They model matrices in linear algebra,
image data, multidimensional data-sets in statistical analyses.

They are found in various more-or-less user friendly library implementations;
the least of which is the regular \texttt{double my\_C\_array[\textit{N}][\textit{M}]} of
C, or the \texttt{real*8, dimension (\textit{N}, \textit{M}) :: my\_F\_array} of Fortran.

More pleasant libraries are found in computer algebra systems: NumPy/SciPy, Matlab,
R, Mathematica, and the like. These are all excellent abstractions over
writing your nested loops by hand, and provide large libraries of useful functions.

There is also the array programming languages, such as APL, J and K, which provide
much the same functionality, but in far fewer keystrokes.

The first and most obvious problems is that none of these libraries are interoperable
at the library level: apart from NumPy taking direct inspiration from Matlab, they
all have different ways to work with the data stored in arrays.

Even C and Fortran don't agree; the address of \texttt{my\_C\_array[0][0]} is next
to \texttt{my\_C\_array[0][1]}, while the same is true of \texttt{my\_F\_array (0, 0)}
and \texttt{my\_F\_array (1, 0)}.

This anarchy of convetion is indicative of the way each of these libraries and
languages came about: ad hoc. Each have been more concerned with practical applications
at every step of their design.

In the absence of theory, many designs have arisen, seemingly with little common
ground. It is reasonable to think there is a unifying theory behind these many associated
models, and that is what we will explore in this chapter.

\section{Arrays}

\begin{definition}
  \label{def:finseq}
  A finite sequence drawn from a set \(S\) is a function \(s : \Nat_{<k} \to S\),
  where \(\Nat_{<k}\) is the set \(\{ n \in \Nat \mid n < k \}\).
\end{definition}

\begin{definition}
  \label{def:array}
  An \(n\)-dimensional array over a set \(S\) generalizes a finite sequence and
  is a function \(A : \Nat_{<k_1} \times \Nat_{<k_2} \times \cdots \times \Nat_{<k_n} \to S\).
  The finite sequence \(k_1, k_2, \dots, k_n\) is called the \emph{shape} of the array, denoted
  \(\sharp A : \Nat_{<n} \to \Nat\), while \(n\) is called the \emph{rank}.
\end{definition}

\begin{observation}
  \label{ob:rank}
  From this defintiion, the rank \(n\) of an array \(A\) can be found by taking the
  shape of the shape, \(\sharp\sharp A\). This technically gives us a finite sequence
  \(f : \Nat_{<1} \to \Nat\), which has the property (and complete definition) that \(f(0) = n\).
\end{observation}

\begin{remark}
  \label{rem:scalar}
  When we deal with arrays that have unit shape, we shall not make special effort
  to notationally distinguish them from scalars when it is clear from context that
  the conversion is implicit.
\end{remark}

\begin{example}
  \label{ex:syntax}
  For the remainder of this report, we shall adopt a familiar syntax for arrays:
  \begin{align*}
    A = [ x_1\; x_2\; \dots \; x_k ] \quad \text{and} \quad
    B = \begin{bmatrix}
      y_{1,1} & \cdots & y_{1,n} \\
      \vdots & \ddots & \vdots \\
      y_{m,1} & \dots  & y_{m,n} 
    \end{bmatrix}
  \end{align*}
  where
  \begin{align*}
    \sharp A = [k] \quad \text{and} \quad \sharp B = [ m\; n ] && .
  \end{align*}
  Indexing of arrays will be denoted with function-call notation, as in Fortran:
  \begin{align*}
    A(n) = x_n \quad \text{and} \quad B(m,n) = y_{m,n} && .
  \end{align*}
  We will also use familiar terminology from linear- and tensor algebra.
\end{example}

\begin{definition}
  \label{def:sheq} 
  We denote that two arrays \(A, B\) are of the same shape \(\sharp A = \sharp B\)
  by writing \(A \sharpeq B\).
\end{definition}

\begin{definition}
  \label{def:reshape-comp} 
  Two array shapes, \(\sharp A\) and \(\sharp B\) are reshape-compatible, iff the product
  of their entries are equal. We denote this relationship by 
  \begin{align*}
    \sharp A \prodeq \sharp B \iff
    \prod_{i} (\sharp A)(i) = \prod_{i} (\sharp B)(i)
  \end{align*}
\end{definition}

\begin{theorem}
  \label{the:reshape-comp-eq}
  Reshape compatibility, \(\sharp A \prodeq \sharp B\) is an equivalence relation. 
\end{theorem}

\begin{proof}
  The definition of reshape-compatibility, is based directly on equality
  on natural numbers, which is trivially an equivalence relation.
\end{proof}

\begin{definition}
  \label{def:flat}
  The ordered traversal of an array \(A : \Nat_{<x_1} \times \cdots \times \Nat_{<x_n} \to S\) is
  a finite sequence \(\flat A : \Nat_{<k} \to S\) where \(k = \prod_i x_i\). This finite sequence
  traverses the original array in lexicographical order over \(\Nat_{<x_1} \times \cdots \times \Nat_{<x_n}\).
\end{definition}

\begin{definition}
  \label{def:reshape-equiv}
  Two arrays are the reshapes of one another when \(\flat A = \flat B\). This of course
  implies reshape-compatibility. We may occasionally write \(A \flateq B\) for short.
\end{definition}

\begin{theorem}
  \label{the:unique-reshape}
  Given an array \(A\) and a shape \(\rho\), where \(\sharp A \prodeq \rho\), there is a unique
  array \(B\) with \(\sharp B = \rho\) and \(A \flateq B\).
\end{theorem}
\begin{proof}
  Suppose there exists \(B_1, B_2\) where \(\sharp B_1 = \sharp B_2 = \rho\)
  and \(B_1 \flateq B_2 \flateq A\) and \(B_1 \neq B_2\).
  
  This implies that there exists \(\brm x = [x_1\; \dots x_n]\) where \(\sharp \brm x = \sharp \rho\),
  such that \(B_1(\brm x) \neq B_2(\brm x)\).

  This implies that there exists \(k\) as per the ordering scheme
  described in definition \ref{def:flat} so that \(B_1(\brm x) = \flat B_1 (k)\) and
  \(B_2(\brm x) = \flat B_2 (k)\) and hence \(\flat B_1(k) \neq \flat B_2(k)\). \contradict
\end{proof}

\begin{definition}
  The reshape function \(\varrho(A, \rho)\) takes an array and a reshape-compatible shape
  to the corresponding unique rehsaped array as per theorem \ref{the:unique-reshape}.
\end{definition}

\section{Categorical setting}

\begin{definition}
  Let the groupoid of products \(\Nat\brm{prod}\) be as follows:
  \begin{itemize}
    \item Every object \(A\) is a finite sequence of natural numbers \(n_1, \dots, n_k\).
    \item There is an isomorphism between two sequences \(A, B\) iff \(\prod_i A_i = \prod_j B_j\)
  \end{itemize}
\end{definition}

\begin{definition}
  We can partially describe the structure of arrays by the functor
  \(\brm{Ary}_1 : \brm{Set} \to \brm{Set} \times \Nat\brm{prod}\) as follows

  \begin{align*}
    \brm{Ary}_1 (A) = (A, 1)
    \brm{Ary}_1 (f) = (f, \mathrm{id})
  \end{align*}
\end{definition}

\begin{theorem}
  \(\brm{Ary}_1\) is free.
\end{theorem}

\chapter{Implementation}

\section{Rust}

\vfill
\begin{center}\itshape End report.\end{center}
\clearpage

\appendix
\renewcommand\thesection{\Alph{section}}
\phantomsection
\addcontentsline{toc}{chapter}{Appendices}

\section{Appendix}

\phantomsection
\addcontentsline{toc}{chapter}{References, Lists \& Index}

\vfill
\begin{center}\itshape End appendices.\end{center}
\clearpage

\printbibliography[heading=subbibintoc]
\listoftables
\listoffigures
\printindex
\end{document}
