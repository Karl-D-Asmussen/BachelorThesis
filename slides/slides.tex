\documentclass[xetex,mathserif,serif]{beamer}

\usepackage{amsmath}
\usepackage{graphicx}
\usepackage{bm}
\usepackage{xcolor}

\newcommand\mrm[1]{\mathrm{#1}}
\newcommand\brm[1]{\bm{\mrm{#1}}}
\newcommand\Nat{\mathbb{N}}
\newcommand\Real{\mathbb{R}}
\newcommand\Com{\mathbb{C}}
\newcommand\Arr[1]{{\brm{Arr}_{\brm{#1}}}}
\newcommand\SFR{\sharp\flat\varrho}
\newcommand\SSFR{\sigma\sharp\flat\varrho}
\newcommand\DSSFR{\div\sigma\sharp\flat\varrho}
\newcommand\ADSSFR{\approx\div\sigma\sharp\flat\varrho}
\newcommand\CADSSFR{c\cdot\approx\div\sigma\sharp\flat\varrho}
\newcommand\XCADSSFR{\times c\cdot\approx\div\sigma\sharp\flat\varrho}
\newcommand\impl{\mathrel{\Rightarrow}}
\newcommand\tlaf{\mathop{\rotatebox[origin=c]{180}{$\flat$}}}
\newcommand\reduce{\operatorname*{\brm{reduce}\,}}
\newcommand\diag{\operatorname*{\brm{diag}\,}}
\newcommand\tile{\operatorname*{\brm{tile}\,}}
\newcommand\oprodbgraphicxy[1]{\mathop{\operatorname*{\,\brm{by}}_{#1}}}

\newcommand\NB[1]{\textcolor{red}{#1}}

\begin{document}

\begin{frame}
  \frametitle{Multidimensional Arrays: Formally Specified by Shape Description and Sliced by Affine Functions}
  \framesubtitle{Bachelor's thesis by Karl D. Asmussen \\
  With advisor Jyrki Katajainen}
  
  \[
    \bm{\mathrm{A}} : m \times n
  \]
  \[ \bm{\mathrm{A}}=
      \begin{bmatrix}
        a_{\text{\tiny$0,0$}} & \cdots & a_{\text{\tiny$0,n{-}1$}} \\
        \vdots & \ddots & \vdots \\
        a_{\text{\tiny$m{-}1,0$}} & \cdots & a_{\text{\tiny$m{-}1,n{-}1$}}
      \end{bmatrix}
  \]
\end{frame}

\begin{frame}
  \frametitle{What is a multidimensional array?}
  \stepcounter{beamerpauses} 
  \begin{itemize}[<+->]
    \item A mapping (function) from integer coordinates to some underlying set
    \item Coordinates taken from higher-dimensional space
    \item Domain restricted to a bounding box with corner at origo (product order)
  \end{itemize}
\end{frame}

\begin{frame}
  \frametitle{The Shape of an array}
  % SHAPE
  \stepcounter{beamerpauses} 
  \begin{itemize}[<+->]
    \item The rank of an array $\brm A$ is the number of dimensions $d$ in its domain space
    \item Product order: $(a, b) < (a', b') \iff a < a' \land b < b'$
    \item Bounding box corners: $\brm o = (0, \dots, 0), \brm x = (k_0, \cdots, k_d)$
    \item Domain: $\{\brm c \in \Nat^d \mid \brm o \le \brm c \le \brm x\}$
    \item $= \Nat_{<k_0} \times \cdots \times \Nat_{<k_d}$
    \item Shape: $\brm x$
    \item ``Sharp:'' $\sharp A = \brm x$
  \end{itemize}
\end{frame}

\begin{frame}
  \frametitle{Shape description}
  % DESC
  \stepcounter{beamerpauses} 
  \begin{itemize}[<+->]
    \item A shape is just a vector of natural numbers
    \item (a rank-1 array, a list)
    \item The number of elements in an array is precisely the
      product of the shape
    \item $|\brm A| = \prod_{i=0}^d \sharp \brm A (i)$
    \item Operations on shape list $\iff$ operations on arrays?
    \item Operations on shape product $\iff$ operations on arrays?
  \end{itemize}
\end{frame}

\begin{frame}
  \frametitle{Correspondece}
  % TRANS
  \stepcounter{beamerpauses} 
  \begin{itemize}[<+->]
    \item Swap elements $[2\ 3] \leadsto [3\ 2]$
    \item Commutativity $2\times3 = 3\times2$
    \item \NB{Transpose} $\brm M : 2\times3\quad \brm M^\intercal : 3\times2$
    \item $
      \begin{bmatrix}
        \NB 1 & 2 & 3 \\
        4 & \NB 5 & 6
      \end{bmatrix} \leadsto
      \begin{bmatrix}
        \NB 1 & 4 \\
        2 & \NB 5 \\
        3 & 6
      \end{bmatrix}
    $
    \item Suggestive!
    \item \NB{Precompose} with coordinate permutation
    \item $\sigma = (x, y) \mapsto (y, x)$
    \item $\sharp \brm A = [2\ 3] \implies \sharp(\brm A \circ \sigma) = [3\;2]$ 
    \item Generalizes
    \item $\sigma' = (x, y, z) \mapsto (z, x, y)$
    \item $\sharp \brm B = [2\ 3\ 4] \implies \sharp(\brm B \circ \sigma') = [4\ 2\ 3]$ 
  \end{itemize}
\end{frame}

\begin{frame}
  \frametitle{Correspondece}
  % TILE
  \stepcounter{beamerpauses} 
  \begin{itemize}[<+->]
    \item Insert $[2] \leadsto [2\;4]$
    \item Multiply by constant $2\leadsto 2\times 4$
    \item \NB{Tile} $
      \begin{bmatrix} 1 \\ 2 \end{bmatrix}\leadsto
      \begin{bmatrix} 1 & \NB 1 & \NB 1 & \NB  1 \\ 2 & \NB 2 & \NB 2 & \NB 2 \end{bmatrix}
    $
    \item Precompose with projection
    \item $\pi = (x, y) \mapsto (x)$
    \item $\sharp \brm A = [2] \implies \NB{\forall n.} \sharp(\brm A \circ \pi) = [2\ \NB n]$
    \item Let $n = 4$
    \item Let $\pi : \Nat_{<2} \times \Nat_{<4} \to \Nat_{<2}$
    \item Generalises
    \item $[2\ 3] \leadsto [2\ 4\ 3]$
    \item $\pi' = (x, y, z) \mapsto (x, z)$
  \end{itemize}
\end{frame}

\begin{frame}
  \frametitle{Correspondece}
  % DIAG
  \begin{itemize}[<+->]
    \item Remove non-unique $[3\ 2\ 2] \leadsto [3\ 2]$
    \item Square free $3\times 2\times 2\leadsto 3 \times 2$
    \item \NB{Diagonal} $
      \begin{bmatrix}
        \NB 1 & 2 \\
        3 & \NB 4
      \end{bmatrix} \leadsto [1\ 4]
      $
    \item Generalizes $
      \begin{bmatrix}
        \begin{pmatrix}
          \NB 1 & 2 \\ 3 &\NB  4
        \end{pmatrix}
        \begin{pmatrix}
          \NB 5 & 6 \\ 7 & \NB 8
        \end{pmatrix}
        \begin{pmatrix}
          \NB 9 & 10 \\ 11 & \NB{12}
        \end{pmatrix}
      \end{bmatrix} = 
      \begin{bmatrix}
        (\NB 1\ 2) & (3\ \NB 4) \\
        (\NB 5\ 6) & (7\ \NB 8) \\
        (\NB 9\ 10) & (11\ \NB{12})
      \end{bmatrix}  
      $
    \item Precompose with ``duplication'' function
    \item $W = (x, y) \mapsto (x, y, y)$
  \end{itemize}
\end{frame}

\begin{frame}
  \frametitle{Correspondece}
  % FLAT
  \stepcounter{beamerpauses} 
  \begin{itemize}[<+->]
    \item Compute product $2 \times 3 = 6$
    \item Reduce by multiplication $\left[\prod_{i=0}^1 [2\ 3](i)\right] = [ 6 ]$
    \item ``Vectorize'' or \NB{Flatten} $
      \begin{bmatrix}
        1 & 2 & 3 \\
        4 & 5 & 6
      \end{bmatrix} \leadsto [1\ 2\ 3\ 4\ 5\ 6]
      $
    \item Precompose with lexicographical order on domain $
      L = \{(0) \mapsto (0,0), (1)\mapsto (0,1), (2)\mapsto(0,2), (3)\mapsto(1,0), (4)\mapsto(1,1), (5)\mapsto(1,2)\}
      $
    \item $\sharp \brm A = [2\ 3] \implies \sharp (\brm A \circ L) = [6] $
    \item \NB{Row} major: C, Numpy
    \item Right associative $\Pi$-type\\
      \texttt{template<int n, typename T> class Array;} \\
      \texttt{typedef Array<2, Array<3, T>> MyArray;}
    \item $\flat \brm A$ to match $\sharp \brm A$
  \end{itemize}
\end{frame}

\begin{frame}
  \frametitle{Correspondece}
  % TLAF
  \begin{itemize}[<+->]
    \item Or? $
      \begin{bmatrix}
        1 & 2 & 3 \\
        4 & 5 & 6
      \end{bmatrix} \leadsto [1\ 4\ 2\ 5\ 3\ 6]
      $ 
    \item Precompose with colexicographical order instead
      $\mrm{co}L = \{ (0)\mapsto(0,0), (1)\mapsto(1,0), (2)\mapsto(0,1), (3)\mapsto(1,1), (4)\mapsto(0,2), (5)\mapsto(1,2) \}$
    \item \NB{Column} major: Fortran
    \item Left associative $\Pi$-type \\
      \texttt{template<typename T, int n> class Array;} \\
      \texttt{typedef Array<Array<T, 2>, 3> MyArray;}
    \item Perhaps $\tlaf \brm A$? Pick one, stick with it.
  \end{itemize}
\end{frame}

\begin{frame}
  \frametitle{Correspondece}
  % RESH
\end{frame}

\end{document}
