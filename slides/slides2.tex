\documentclass[xetex,mathserif,serif]{beamer}

\usepackage{amsmath}
\usepackage{graphicx}
\usepackage{bm}
\usepackage{xcolor}

\usepackage{tikz-cd}

\newcommand\mrm[1]{\mathrm{#1}}
\newcommand\brm[1]{\bm{\mrm{#1}}}
\newcommand\Nat{\mathbb{N}}
\newcommand\Zheil{\mathbb{Z}}
\newcommand\Real{\mathbb{R}}
\newcommand\Com{\mathbb{C}}
\newcommand\Arr[1]{{\brm{Arr}_{\brm{#1}}}}
\newcommand\SFR{\sharp\flat\varrho}
\newcommand\SSFR{\sigma\sharp\flat\varrho}
\newcommand\DSSFR{\div\sigma\sharp\flat\varrho}
\newcommand\ADSSFR{\approx\div\sigma\sharp\flat\varrho}
\newcommand\CADSSFR{c\cdot\approx\div\sigma\sharp\flat\varrho}
\newcommand\XCADSSFR{\times c\cdot\approx\div\sigma\sharp\flat\varrho}
\newcommand\impl{\mathrel{\Rightarrow}}
\newcommand\tlaf{\mathop{\rotatebox[origin=c]{180}{$\flat$}}}
\newcommand\reduce{\operatorname*{\brm{reduce}\,}}
\newcommand\diag{\operatorname*{\brm{diag}\,}}
\newcommand\tile{\operatorname*{\brm{tile}\,}}
\newcommand\oprodby[1]{\mathop{\operatorname*{\,\brm{by}}_{#1}\,}}

\newcommand\NB[1]{\textcolor{red}{#1}}

\begin{document}

\begin{frame}
  \frametitle{Multidimensional Arrays: Formally Specified by Shape Description and Sliced by Affine Functions}
  \framesubtitle{Bachelor's thesis by Karl D. Asmussen \\
  With advisor Jyrki Katajainen}
  
  \[
    \bm{\mathrm{A}} : m \times n
  \]
  \[ \bm{\mathrm{A}}=
      \begin{bmatrix}
        a_{\text{\tiny$0,0$}} & \cdots & a_{\text{\tiny$0,n{-}1$}} \\
        \vdots & \ddots & \vdots \\
        a_{\text{\tiny$m{-}1,0$}} & \cdots & a_{\text{\tiny$m{-}1,n{-}1$}}
      \end{bmatrix}
  \]
\end{frame}

\begin{frame}
  \frametitle{What is a multidimensional array?}
  \pause
  \begin{itemize}[<+->]
    \item A mapping (function) from integer coordinates to some underlying set
    \item Coordinates taken from higher-dimensional space
    \item Domain restricted to a bounding box with corner at origo (product order)
  \end{itemize}
\end{frame}

\begin{frame}
  \frametitle{Terminology}
  \pause
  \begin{itemize}[<+->]
    \item The rank of an array is the number of dimensions \(d\) in its domain space
    \item Array lookup is done at a coordinate \(\brm c = (c_0,\dots,c_d)\)
    \item Coordinates are composed of indexes \(c_0,\dots,c_d\)
    \item The set of values an index can range over \(c_0 \in \Nat_{<k}\) is an axis
    \item \(\Nat_{<k} = \{ n \in \Nat \mid n < k \}\)
    \item The Shape of an array\dots
  \end{itemize}
\end{frame}

\begin{frame}
  \frametitle{The Shape of an array}
  % SHAPE
  \pause
  \begin{itemize}[<+->]
    \item Product order:\\
      \((a, b) < (a', b') \iff a < a' \land b < b'\)\\
      \((a, b) \le (a', b') \iff a \le a' \land b \le b'\)
    \item Bounding box corners: \(\brm o = (0, \dots, 0), \brm x = (k_0, \cdots, k_d)\)
    \item Domain: \(\{\brm c \in \Zheil^d \mid \brm o \le \brm c < \brm x\}\)
    \item \(= \Nat_{<k_0} \times \cdots \times \Nat_{<k_d}\)
    \item Shape: \(\brm x\)
    \item ``Sharp:'' \(\sharp A = \brm x\)
  \end{itemize}
\end{frame}

\begin{frame}
  \frametitle{Shape description}
  % DESC
  \pause
  \begin{itemize}[<+->]
    \item A shape is just a vector of natural numbers
    \item (a rank-1 array, a list)
    \item The number of elements in an array is precisely the
      product of the shape
    \item \(|\brm A| = \prod_{i=0}^d \sharp \brm A (i)\)
    \item Operations on shape list \(\iff\) operations on arrays?
    \item Operations on shape product \(\iff\) operations on arrays?
    \item Correspondence
  \end{itemize}
\end{frame}

\begin{frame}
  \frametitle{Correspondece: Motivating Example --- Transpose}
  % TRANS
  \pause
  \begin{itemize}[<+->]
    \item Permute \([2\ 3] \leadsto [3\ 2]\)
    \item Commutativity \(2\times3 = 3\times2\)
    \item \NB{Transpose} \(\brm M : 2\times3\quad \brm M^\intercal : 3\times2\)
    \item \(
      \begin{bmatrix}
        \NB 1 & 2 & 3 \\
        4 & \NB 5 & 6
      \end{bmatrix} \leadsto
      \begin{bmatrix}
        \NB 1 & 4 \\
        2 & \NB 5 \\
        3 & 6
      \end{bmatrix}
      \)
    \item Suggestive!
    \item \NB{Precompose} with coordinate permutation
    \item \(\sigma = (x, y) \mapsto (y, x)\)
    \item \(\sharp \brm A = [2\ 3] \implies \sharp(\brm A \circ \sigma) = [3\;2]\) 
    \item Generalizes
    \item \(\sigma' = (x, y, z) \mapsto (z, x, y)\)
    \item \(\sharp \brm B = [2\ 3\ 4] \implies \sharp(\brm B \circ \sigma') = [4\ 2\ 3]\) 
  \end{itemize}
\end{frame}

\begin{frame}
  \frametitle{Correspondece: More in the Same Vein --- Tile}
  % TILE
  \pause
  \begin{itemize}[<+->]
    \item Insert \([2] \leadsto [2\;4]\)
    \item Multiply by constant \(2\leadsto 2\times 4\)
    \item \NB{Tile} \[
      [1\;2]\leadsto
      \begin{bmatrix} 1 \\ 2 \end{bmatrix}\leadsto
      \begin{bmatrix} 1 & \NB 1 & \NB 1 & \NB  1 \\ 2 & \NB 2 & \NB 2 & \NB 2 \end{bmatrix}
    \]
    \item Precompose with projection
    \item \(\pi = (x, y) \mapsto (x)\)
    \item \(\sharp \brm A = [2] \implies \NB{\forall n.} \sharp(\brm A \circ \pi) = [2\ \NB n]\)
    \item NumPy broadcasting
    \item Let \(n = 4\)
    \item Let \(\pi : \Nat_{<2} \times \Nat_{<4} \to \Nat_{<2}\)
    \item Generalises
    \item \([2\ 3] \leadsto [2\ 4\ 3]\)
    \item \(\pi' = (x, y, z) \mapsto (x, z)\)
  \end{itemize}
\end{frame}

\begin{frame}
  \frametitle{Correspondece: Diagonal}
  % DIAG
  \pause
  \begin{itemize}[<+->]
    \item Remove non-unique \([3\ 2\ 2] \leadsto [3\ 2]\)
    \item Square free \(3\times 2\times 2\leadsto 3 \times 2\)
    \item \NB{Diagonal} \(
      \begin{bmatrix}
        \NB 1 & 2 \\
        3 & \NB 4
      \end{bmatrix} \leadsto [1\ 4]
      \)
    \item Generalizes \[
      \!\!\!\!
      \begin{bmatrix}
        \begin{pmatrix}
          \NB 1 & 2 \\ 3 &\NB  4
        \end{pmatrix}
        \begin{pmatrix}
          \NB 5 & 6 \\ 7 & \NB 8
        \end{pmatrix}
        \begin{pmatrix}
          \NB 9 & 10 \\ 11 & \NB{12}
        \end{pmatrix}
      \end{bmatrix}_3 = 
      \begin{bmatrix}
        (\NB 1\ 2) & (3\ \NB 4) \\
        (\NB 5\ 6) & (7\ \NB 8) \\
        (\NB 9\ 10) & (11\ \NB{12})
      \end{bmatrix} \leadsto
      \begin{bmatrix}
        1 & 4 \\
        5 & 8 \\
        9 & 12 
      \end{bmatrix}
      \]
    \item Precompose with ``duplication'' function
    \item \(W = (x, y) \mapsto (x, y, y)\)
  \end{itemize}
\end{frame}

\begin{frame}
  \frametitle{Correspondece: Indexing Generalized}
  % FIXATE
  \pause
  \begin{itemize}[<+->]
    \item Delete \([2\;3] \leadsto [2]\)
    \item Index/extract/project subarray/row/column
      \[
        \begin{bmatrix}
          1 & \NB 2 & 3 \\ 4 & \NB 5 & 6
        \end{bmatrix} \leadsto
        [2\;5]
      \]
    \item Corner case: delete \([2] \leadsto []\)
    \item Zero-dimensional arrays have 1 element: \([1\;\NB 2\;3] \leadsto [2]_0\)
    \item Precompose with constant function
    \item \(k = (x)\mapsto (x, 2)\)
  \end{itemize}
\end{frame}

\begin{frame}
  \frametitle{Correspondece: Slice}
  % SLICE
  \pause
  \begin{itemize}[<+->]
    \item Vector subtraction \( [3\;5] - [1\;2] = [2\;3] \)
    \item Slice with stride
      \[
        \begin{bmatrix}
          1 & \NB 2 & \NB 3 & \NB 4 & 5 \\
          6 & 7 & 8 &  9 & 10 \\
          11 & \NB{12} & \NB{13} & \NB{14} & 15
        \end{bmatrix} \leadsto \begin{bmatrix}
          2 & 3 & 4 \\
          12 & 13 & 14
        \end{bmatrix}
      \]
    \item Precompose with affine functions of each axis \(f(x) = ax + b\)
    \item \(S = (x, y) \mapsto (2x + 0, 1y + 1)\)
    \item \(S : \Nat_{<3}\times\Nat_{<5} \to \Nat_{<2}\times\Nat_{<3}\)
  \end{itemize}
\end{frame}


\begin{frame}
  \frametitle{Correspondece: Key insight}
  \pause
  \begin{itemize}[<+->]
    \item Transpose, Diagonal, Tile, Fixate, Slice.
    \item Can be modeled as linear transforms in homogenous coordinates.
    \item Transpose \(\bigg(\begin{smallmatrix} 1 & 0 & 0 && 0 \\ 0 & 0 & 1 && 0 \\ 0 & 1 & 0 && 0 \end{smallmatrix}\bigg)\)
    \item Diagonal \(\big(\begin{smallmatrix} 1 & 0 & 0 && 0 \\ 0 & 1 & 1 && 0 \end{smallmatrix}\big)\)
    \item Tile \(\big(\begin{smallmatrix} 1 & 0 & 0 && 0 \\ 0 & 1 & 0 && 0 \end{smallmatrix}\big)\)
    \item Fixate \(\bigg(\begin{smallmatrix} 1 & 0 && 0 \\ 0 & 0 && 1 \\ 0 & 1 && 0 \end{smallmatrix}\bigg)\)
    \item Slice \(\big(\begin{smallmatrix} 1 & 0 && 1 \\ 0 & 2 && 0 \end{smallmatrix}\big)\)
    \item Bonus: Reverse \(\big(\begin{smallmatrix} -1 & 0 && k_0-1 \\ 0 & 1 && 0 \end{smallmatrix}\big)\)
  \end{itemize}
\end{frame}

\begin{frame}
  \frametitle{Correspondece: Summary so far \#1}
  \pause
  \begin{itemize}[<+->]
    \item Precompose with linear or affine function in natural coordinates \(f : \Nat^q \to \Nat^d\).
    \item Linear: transpose, diagonal, tile, reverse
    \item Preserves origin index \(A(0, \dots, 0) = (A \circ f)(0, \dots, 0)\)
    \item Affine: slice, fixate
  \end{itemize}
\end{frame}

\begin{frame}
  \frametitle{Correspondece: Flat}
  % FLAT
  \pause
  \begin{itemize}[<+->]
    \item Compute product \(2 \times 3 = 6\)
    \item Reduce by multiplication \(\left[\prod_{i=0}^1 [2\ 3](i)\right] = [ 6 ]\)
    \item ``Vectorize'' or \NB{Flatten} \[
      \begin{bmatrix}
        1 & 2 & 3 \\
        4 & 5 & 6
      \end{bmatrix} \leadsto [1\ 2\ 3\ 4\ 5\ 6]
      \]
    \item Precompose with lexicographical order on domain $
      L = \{(0) \mapsto (0,0), (1)\mapsto (0,1), (2)\mapsto(0,2), (3)\mapsto(1,0), (4)\mapsto(1,1), (5)\mapsto(1,2)\}
      $
    \item \(\sharp \brm A = [2\ 3] \implies \sharp (\brm A \circ L) = [6] \)
    \item \NB{Row} major: C, Numpy
    \item Right associative \(\Pi\)-type\\
      \texttt{typedef Array<\NB 2, Array<\NB 3, T>> MyArray;}
    \item \(\flat \brm A\) to match \(\sharp \brm A\)
  \end{itemize}
\end{frame}

\begin{frame}
  \frametitle{Correspondece: Flat 2}
  % TLAF
  \pause
  \begin{itemize}[<+->]
    \item Or? $
      \begin{bmatrix}
        1 & 2 & 3 \\
        4 & 5 & 6
      \end{bmatrix} \leadsto [1\ 4\ 2\ 5\ 3\ 6]
      $ 
    \item Precompose with colexicographical order instead
      \(\mrm{co}L = \{ (0)\mapsto(0,0), (1)\mapsto(1,0), (2)\mapsto(0,1), (3)\mapsto(1,1), (4)\mapsto(0,2), (5)\mapsto(1,2) \}\)
    \item \NB{Column} major: Fortran
    \item Left associative \(\Pi\)-type \\
      \texttt{typedef Array<Array<T, \NB 2>, \NB 3> MyArray;}
    \item Perhaps \(\tlaf \brm A\)?
    \item Pick one, stick with it.
  \end{itemize}
\end{frame}

\begin{frame}
  \frametitle{Correspondece: Upshape} 
  % UNFLAT
  \pause
  \begin{itemize}[<+->]
    \item Factoring \(2412 = 12 \times 201\) (not neccesarily prime)
    \item Have \(20\times 5\) array \(A\) --- 100 elements.
    \item Desired: \(10 \times 10\) --- 100 elements.
    \item Row major \(\implies\) column stride is 10, row stride is 1.
    \item \(r = (c, r)\mapsto (10c + 1r), r : {\Nat_{<10}}^2 \to \Nat_{<100}\)
    \item \(B = \flat A \circ r, \sharp B = [10\;10]\)
    \item Convert to \NB{mixed radix}
  \end{itemize}
\end{frame}


\begin{frame}
  \frametitle{Correspondece: Reshape}
  % RESH
  \pause
  \begin{itemize}[<+->]
    \item Re-factoring \(4\times 10 \times 9 = 8 \times 15 \times 3\)
    \item Flatten, then un-flatten
    \item Mixed radix conversion in general
    \item From any shape to any shape: \NB{Reshape}
    \item \(\varrho(A, [10\;10])\) as in APL
    \item Flat and Upshape both special cases
  \end{itemize}
\end{frame}

\begin{frame}
  \frametitle{Correspondence: Summary so far \#2}
  \pause
  \begin{itemize}[<+->]
    \item Flatten, Upshape, Reshape
    \item Row major or column major? ``Co-''relationship of some kind.
    \item Two kinds of operations on coordinates: matrix multiplication, mixed radix conversion
    \item Mixed radix conversion also preserves origin index
    \item Lexicographic order not linear --- radix conversion is!
    \item \(5\times3\times12\) becomes \((36\;\;12\;\;1) : 1 \times 3\)
  \end{itemize}
\end{frame}

\begin{frame}
  \frametitle{Correspondece: Map}
  % CAT
  \pause
  \begin{itemize}[<+->]
    \item Identity \([2\;3] \leadsto [2\;3]\)
    \item Apply function to each entry, nothing controversial\\
      \([1\;2\;3] \leadsto [\mrm A\;\mrm B\;\mrm C]\)
    \item \NB{Postcompose}, \(f \circ \brm A\)
    \item Changes type --- very versatile
    \item Arrays of a given shape is a free functor.
  \end{itemize}
\end{frame}

\begin{frame}
  \frametitle{Correspondece: Outer product}
  % PROD
  \pause
  \begin{itemize}[<+->]
    \item Concatenation \([2\;3], [3\;4] \leadsto [2\;3\;3\;4]\)
    \item Multiplication \(2\times3 \;\times\; 3\times4\)
    \item Split the entry coordinate, fetch from two arrays, pair up
      entries
      \[[1\;2\;3] \oprodby{} [4\;5\;6] \leadsto
      \begin{bmatrix}
        \langle1,4\rangle & \langle1,5\rangle & \langle1,6\rangle \\
        \langle2,4\rangle & \langle2,5\rangle & \langle2,6\rangle \\
        \langle3,4\rangle & \langle3,5\rangle & \langle3,6\rangle
      \end{bmatrix}\]
    \item Makes very big arrays
    \item Pair with diagonal to get zip
    \item Is a categorical product
    \item To project into original arrays, fixate axes not in original
      arrays and map with pair projection
    \item Map with two-argument functions
  \end{itemize}
\end{frame}

\begin{frame}
  \frametitle{Correspondece: Reduce}
  % REDUCE
  \pause
  \begin{itemize}[<+->]
    \item Deletion \([2\;4\;3] \leadsto [2\;3]\)
    \item Divide by constant \(\frac{2\times4\times3}4 = 2\times3\)
    \item Collapse along an axis with binary operation
      \[
        \begin{bmatrix}
          1 & 2 & 3 \\
          4 & 5 & 6
        \end{bmatrix} \leadsto
        \begin{bmatrix}
          (1 + 2 + 3) \\
          (4 + 5 + 6) \\
        \end{bmatrix}
      \]
    \item More ``pure'' than fixate: fixate has many ways,
      reduce has only one.
    \item Monoid $>$ semigroup $>$ magma/accumulator function
    \item Free monoid (list/rank-1 array)
  \end{itemize}
\end{frame}

\begin{frame}
  \frametitle{Correspondece: Subarray}
  % REDUCE
  \pause
  \begin{itemize}[<+->]
    \item Sub-sequence extraction \([2\;4\;5\;3\;6] \leadsto [2\;4\;3], [5\;6]\)
    \item Factor? \(2\times4\times5\times3\times6 = (2\times4\times3)\times(5\times6)\)
    \item Create array-of-arrays
      \[
        \begin{bmatrix}
          1 & 2 & 3 \\
          4 & 5 & 6
        \end{bmatrix} \leadsto \big[[1\;4]\;\;[2\;5]\;\;[3\;6]\big]
      \]
    \item Loosely related to reduce by free monoid
    \item Exponential object in category?
    \item New! Figured this one out after Report was done!
  \end{itemize}
\end{frame}

\begin{frame}
  \frametitle{Correspondece: Concatenate}
  % CAT
  \pause
  \begin{itemize}[<+->]
    \item Tensor addition \([2\;2] \oplus [2\;3] = [2\;\;(2+3)]\)
    \item Concatenation
      \[
        \begin{bmatrix} 1 & 2 \\ 3 & 4 \end{bmatrix} +\!\!\!+
        \begin{bmatrix} 5 & 6 & 7 \\ 8 & 9 & 10 \end{bmatrix}
        = \begin{bmatrix}
          1 & 2 & 5 & 6 & 7 \\
          3 & 4 & 8 & 9 & 10
        \end{bmatrix}
      \]
    \item Sort of odd-one-out?
    \item On rank-1 arrays, becomes free monoid operation
    \item Rank-0 arrays (scalars) cannot be concatenated!
    \item Rank-0 arrays trivially map to rank-1 arrays of shape \([1]\)
    \item Also brand new!
  \end{itemize}
\end{frame}

\begin{frame}
  \frametitle{Correspondece: Summary so far \#3}
  % CAT
  \pause
  \begin{itemize}[<+->]
    \item Operations that change the domain
    \item Map, Reduce, Subarray, Outer Product
    \item Odd-one-out: Concatenate
  \end{itemize}
\end{frame}

\begin{frame}
  \frametitle{Overview}
  \pause
  \begin{itemize}[<+->]
    \item Base
      \begin{itemize}[<+->]
        \item Shape: domain axes' upper bounds
        \item Index: function application
      \end{itemize}
    \item Precompose
      \begin{itemize}[<+->]
        \item Linear map: Transpose, Tile, Diagonal, ``Upshape from flat''
        \item Affine map: Slice, Fixate
        \item Monotonous map, (co)lexicographic order: Flat, Reshape
      \end{itemize}
    \item Postcompose: Map
    \item Transformation:
      \begin{itemize}[<+->]
        \item Einstein notation: Reduce
        \item Combinatorical: Outer Product
        \item Select based on index: Concatenate
        \item Enable currying: Subarray
      \end{itemize}
  \end{itemize}
\end{frame}

\begin{frame}
  \frametitle{Futher grouping}
  \pause
  \begin{itemize}[<+->]
    \item Preserves origin index \(\brm A (0,\dots,0) = F(\brm A)(0,\dots,0)\):
    \item Transpose, Diagonal, Tile, Flat, Reshape, Reduce (sorta),
      Outer Product (kinda), Subarray (ish), Concatenate (left)
    \item Representable as matrix:
    \item Transpose, Tile, Diagnoal, Upshape, Slice, Fixate
  \end{itemize}
\end{frame}

\begin{frame}
  \frametitle{Categorical ground}
  \pause
  \begin{itemize}[<+->]
    \item The Arrow category on \(\brm{Set}\)
    \item Not quite: Reduce \& friends.
    \item Not a ``set with structure.''
    \item \emph{Much} work to be done (for me.)
  \end{itemize}
\end{frame}

\begin{frame}
  \frametitle{Diagram chasing}
  % DIAGRAM
  \pause
  \begin{center}
    \begin{tikzcd}[ampersand replacement=\&]
      \Nat_{<k_0}\times\cdots\times\Nat_{<k_d} \ar[r, "A"] \ar[d, "\alpha"] \& S \ar[d, "\omega"] \\
      \Nat_{<j_0}\times\cdots\times\Nat_{<j_e} \ar[r, "B"] \& T 
    \end{tikzcd}\\[1em]
    \pause

    \begin{tikzcd}[ampersand replacement=\&]
      \Nat_{<k_0}\times\cdots\times\Nat_{<k_d} \ar[r, "A"{name=UPPER}] \& S \\
      \Nat_{<j_0}\times\cdots\times\Nat_{<j_e} \ar[r, "B"{name=LOWER}] \& T
      \ar[from=UPPER, to=LOWER, double, shorten <= 4pt, "\Theta"]
    \end{tikzcd}\\[1em]

  \end{center}

  \pause $\alpha$: Transpose, Tile, Diagonal, Slice, Fixate, Reverse

  \pause $\omega$: Map

  \pause $\Theta$: Reduce, Subarray

  \pause Not shown: Outer product, Concatenate
\end{frame}

\begin{frame}
  \frametitle{Practical part}
  \pause
  \begin{itemize}[<+->]
    \item Rust --- the language of the future
    \item No garbage collection --- no problem
    \item Make pointers safe again
    \item Numpy is not fast, not well-founded, solve both!
    \item Inspired by NumPy: lexicographical order
    \item Inspired by NumPy: builtin types are arrays too
    \item As general as possible! (Probably bad idea)
  \end{itemize}
\end{frame}

\begin{frame}
  \frametitle{Code: Design}
  \pause
  \begin{itemize}[<+->]
    \item Traits for reference/mutable reference/owned data
    \item \texttt{slice} and \texttt{mut\_slice} and \texttt{into\_slice}
    \item 10+ methods and assocaited types each --- a lot of work
    \item Generic slice type (matrix)
    \item Generic reshape type (mixed radix conversion)
    \item Separate lazy types for the rest
    \item Implementation for Rust's sliceable types
  \end{itemize}
\end{frame}

\begin{frame}
  \frametitle{Code: Hard Truths}
  \pause
  \begin{itemize}[<+->]
    \item I am not as proficient in Rust as I'd like to be
    \item Return reference or value?
    \item If reference, then what about lazy arrays?
    \item Rust lacks Higher Kinded Polymorphism
    \item \texttt{fn function<'lifetime>(\&'lifetime reference\_arg) -> \&'lifetime ResultType;}
    \item Output type is parameterized on lifetime: no Associated HKP, no cookie
    \item \texttt{trait Polymorphism \{\\
            \ \ \ type LTPAssocType<'lifetime> = \&'lifetime i32;\\
                  \}}
    \item It compiles! It uses unsafe type casting! Unsound!
  \end{itemize}
\end{frame}

\begin{frame}
  \frametitle{Code: futher work}
  \pause
  \begin{itemize}[<+->]
    \item Hack to simulate lifetime parameterization
    \item Trait with associated type that has void reference of same lifetime
    \item \texttt{trait LT \{ type Void; \}\\
      impl<'a, T> LT for \&'a T \{ type Void = \&'a (); \}}
    \item Trait with associated type that has the lifetime replaced
    \item \texttt{trait LTP<'a> \{ type Rep; \}\\
      impl<'a, 'b, T> LTP<'b> for \&'b T \\
      where Self : LP<Void=\&'a ()>\\
      \{ type Rep = \&'a T; \}}
    \item Return reference or value? Fake reference type
    \item Behaves like reference, has \texttt{into\_inner} method
  \end{itemize}
\end{frame}

\begin{frame}
  \frametitle{Conclusion}
  \pause
  \begin{itemize}[<+->]
    \item Not trivial!
    \item Not non-trivial!
    \item Merits futher study
    \item At least two papers in the works
    \item Master thesis: the same but in Idris!
  \end{itemize}
\end{frame}

\end{document}
